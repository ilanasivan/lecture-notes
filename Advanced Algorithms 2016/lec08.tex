\documentclass[a4paper]{article}

\usepackage[T5]{fontenc}
\usepackage[utf8]{inputenc}
\usepackage{amsfonts,amssymb}
\usepackage{mathtools}
\usepackage[iso]{datetime}
\usepackage{tabu}
\usepackage[colorlinks=true,urlcolor=blue,linkcolor=black]{hyperref}
\usepackage{tikz}

\title{Advanced Algorithms\\\large Lecture 8}
\date{2016-12-18 \\ Last edited \currenttime\ \today}
\author{Lecture by Shay Mozes\\Typeset by Steven Karas}

\newenvironment{itemize*}%
  {\begin{itemize}%
    \setlength{\itemsep}{0pt}%
    \setlength{\parsep}{0pt}%
    \setlength{\parskip}{0pt}}%
  {\end{itemize}}

\newenvironment{enumerate*}%
  {\begin{enumerate}%
    \setlength{\itemsep}{0.5pt}%
    \setlength{\parsep}{0pt}%
    \setlength{\parskip}{0pt}}%
  {\end{enumerate}}

\begin{document}

\maketitle

\paragraph{Notes}

Mid-term feedback survey is available on Moodle site.

Final exam will follow format from previous years.
Last years in particular is good.
The first year he taught the course the exam was a little too long.

\section{Linear Programming}

\paragraph{Recommended reading}
\begin{itemize*}
  \item CLRS (from 2nd edition onwards)
  \item "Linear Algebra and Its Applications", by Gilbert Strang, chapter 8
  \item "Linear Programming", by Vasek Chvatal
  \item "Introduction to Linear Optimization", by Dimitris Bertsimas and John Tsitsiklis
  \item \href{http://www.sce.carleton.ca/faculty/chinneck/po.html}{Lecture Notes} by John W. Chinneck
\end{itemize*}

While the term "Programming" is largely historical, it is one of the most widely used applications of computer science.
At its core, linear programming is optimizing a system of linear inequalities (called constraints).

\paragraph{Example: Diet Problem}
As a poor student, we want to spend as little money as possible on food. However, we need to also eat enough to stay alive and healthy.

In a proper diet, we need to eat 4 units of protein a day. Peanut butter provides 1 unit, and costs 2\$. Steak provides 2 units, and costs 3\$.

Let $x=$ the amount of peanut butter per day in the diet.
Let $y=$ the amount of steak per day in the diet.

So we want to minimize $2x+3y$ subject to the constaints $x+2y \ge 4$, $x \ge 0$, and $y \ge 0$.

\paragraph{Example applications}

\begin{itemize*}
  \item Airline crew scheduling
  \item Manufacturing and production planning
  \item Telecommunications network design
\end{itemize*}

\paragraph{Formal}
A linear program is a problem with $n$ variables $x_1,...,x_n$, an object function of the form $\min c_1x_1+...+c_nx_n$ (or $\max$), and a set of $m$ linear constraints of the form $a_{i1}x_1+...+a_{in}x_n <=> b_i$, along with the coefficients $a$, $b$, and $c$.

\paragraph{Matrix form}

$\max c^\intercal x$ such that $Ax \le b$ where $x$ is a vector of $n$ variables, $c$ is a vector of $n$ objective function coefficients, $A$ is an $n$ by $m$ matrix of constraint coefficients, and $b$ is a vector of $m$ constraint values.

\paragraph{Geometric Intuition}
Each constraint subdivides the solution space in 2.
Feasible solutions satisfy all the constraints.
The Feasible Set is the set of all feasible solutions.
Another way to define this is as the intersection of all the halfspaces where each constraint is satisfied.
Any intersection of halfspaces is called a convex polyhedron, so therefore the feasible set is a convex polyhedron.

A bounded and nonempty polyhedron is called a polytope.

Generally speaking, we have three cases:

\begin{enumerate*}
  \item The feasible set is empty (the problem is not feasible)
  \item The feasible set is unbound (as in the above example)
  \item The feasible set is bounded and nonempty (a polytope)
\end{enumerate*}

The first two cases are very rare for practical purposes.

\subparagraph{Fact:}
every point $p$ in a convex polytope can be represented as a convex combination of the vertices $v_i$ of the polytope.

\[p=\sum \lambda_i v_i \quad (0 \le \lambda_i \le 1 \;;\; \sum \lambda_i = 1)\]

\subsection{Optimal values are always at a vertex}

The objective function defines a family of parallel hyperplanes.
We want to find one that intersects the feasible set and minimizes the objective function.

\paragraph{Proof}
Let $p$ be a point in the feasible set.
Write $p=\sum \lambda_i v_i$.
By linearity of the objective function $z$, $z(p)=\sum \lambda_i z(v_i) \le z(v_{\max})$ where $v_{\max}$ is the vertex that maximizes $z$.

\subsection{Standard Form}
Maximize $\sum\limits_{j=1}^n c_j x_j$ subject to $\begin{matrix}\sum_{j=1}^n a_{ij} x_j & i=1 ... m \\ x_j\ge 0 & j = 1 ... n\end{matrix}$.

\paragraph{Standard matrix form:}
Maximize $C^\intercal x$ such that $Ax \le b$ and $x \ge 0$.

\subsection{Solving LP}
\begin{itemize*}
  \item Simplex method. Non-polynomial, but fast for practical purposes.
  \item Ellipsoid method. First polynomial solution from 80's.
  \item Interior Point method. Modern methods are comptetive with Simplex.
\end{itemize*}

\subsection{Simplex Algorithm}
Assign additional "slack" variables, that together provide a computational way of iterating over the vertices of the feasible set.

\paragraph{Example: Bakery}
Bob's bakery sells bagels and muffins.
To bake a dozen bagels, Bob needs 5 cups of flour, 2 eggs, and one cup of sugar.
To bake a dozen muffins, Bob needs 4 cups of flour, 4 eggs, and two cups of sugar.
Bob can sell bagels for \$10/dozen and muffins for \$12/dozen.
Bob has 50 cups of flour, 30 eggs, 20 cups of sugar.

How many bagels and muffins should Bob make to maximize his revenue?

\subparagraph{Formally}

\begin{tabu} to \linewidth {c|ccc}
& Bagels & Muffins & Available \\
\hline
Flour & 5 & 4 & 50 \\
Eggs & 2 & 4 & 30 \\
Sugar & 1 & 2 & 20
\end{tabu}

\[
A=\begin{bmatrix}
5 & 4 \\
2 & 4 \\
1 & 2
\end{bmatrix}
\quad
b=\begin{bmatrix}
50 \\
30 \\
20
\end{bmatrix}
\quad
c=\begin{bmatrix}
10 \\
12
\end{bmatrix}
\]

\paragraph{Exercise: Max flow}
Given a directed graph $G=(V,E)$ with non-negative arc capacities.

Let the objective function and the constraint matrix be the weighted adjacency matrix $E$.

\subparagraph{}
$f_e$ is the flow of the edge $e$, where $e=uv$

\[\max \sum_{e:head(e)=t} f_e\]

% capacity
\[\forall e\in E : f_e \le c_e \]
% leakage
\[\forall v\in V \setminus \{s,t\} : \sum_{e:head(e)=v} f_e - \sum_{e:tail(e)=v} f_e = 0\]

Overall, we have $m+n-2$ constraints.

\paragraph{Example: Toy Factory}
A toy factory produces dolls and cars.
A new employee called Danny is hired.
He can produce 2 cars and 3 dolls each day.
However, the packaging machine can only pack 4 items each day.
The company's profit is \$15 per car and \$10 per doll.

Note that we know that the global optimum must be at a vertex of the polytope.
We can also prove that a local optimum is also a global optimum.

\paragraph{Simplex Algorithm}
In a nutshell, move between neighboring vertices towards a local optimum.\footnote{The pathalogical case is where the objective function is parallel to one of the faces of the polytope. These cases are covered in the recommended reading.}

\begin{enumerate*}
  \item Find any vertex as a starting point. The origin can be used as a starting point in many cases.
  \item Repeatedly move to a neighboring vertex that improves the objective function.
\end{enumerate*}

\paragraph{Example: Toy Factory}
Start at $(0,0)$ with an objective value of $z(0,0)=0$.
Take the first step to $(2,0)$ with an objective value of $z(2,0)=30$.
Take the second step to $(2,2)$ with an objective value of $z(2,2)=50$.
Check the only neighboring vertex at $(1,3)$, but it's objective value is $z(1,3)=45 < 50$.
Therefore, the optimum is $(2,2)$ with an objective value of $50$.

\paragraph{Finding vertices algebraically}
At a vertex a subset of the inequalities are equalities.
It is easy to find the intersection of linear equalities (solution to a system of equations).
We will add "slack" variables to determine which inequalities are "tight" or not.

\[\binom{n+m}{n} = \frac{(n+m)!}{n!m!}\]

\begin{description}
  \item[Nonbasic variable] a variable currently set to zero by the simplex method.
  \item[Basic variable] a variable that is not currently set to zero by the simplex method.
    The values of basic variables are determined by the nonbasic variables.
  \item[A basis] the current set of basic variables
\end{description}

In each step, the basic variable that improves the objective function the fastest replaces the nonbasic variables that improves the objective function the least with regards to the entering variable. 

%TODO: copy the example run of the simplex algorithm

\paragraph{Summary: Simplex Algorithm}
Not polynomial, but typically runs quite well.
We also assume that the matrix A is non-singular (otherwise omitting linearly dependent constraints).
Degenerate steps don't improve the objective function.

\subsection{Integer Programming}
Linear programming with an additional constraint that the values can only ever be integers.

\subsubsection{Binary Integer Programming}
Values can only ever be 0 or 1.

NP-Hard, but can be used to model many problems.

\paragraph{Example: Vertex Cover}
Variables are for each $v \in V$, $x_v$ indicates if $v$ is in the cover.
Subject to: $\forall i,j \in E : x_i + x_j \ge 1$, $x_v \in \{0, 1\}$.
Minimize $\sum_v x_v$.

\paragraph{Example: Set Cover}
Given the subsets $S_1,...,S_n$ of $\{1,...,m\}$ and a cost $p_i$ for each subset $S_i$; Minimize the total cost of subsets that cover the entire set.

Let $x_i \in \{0, 1\}$ be whether or not the subset $S_i$ is included in the cover. Minimize $\sum_i p_i \cdot x_i$ subject to the constraint that $\forall i \in \{1,...,m\} : \sum_{i:j \in S_i} x_i \ge 1$

\paragraph{Example: Shortest Path}
Given a directed graph $G=(V,E)$ and $s,t \in V$ and a nonnegative length $p_e$ for each edge $e \in E$.
Let the variables be for each edge $x_e$ which indicates if $e$ is in the path.

Minimize $\sum_e p_e \cdot x_e$ subject to $x_e \in \{0, 1\}$ and:

\[\forall u\in V \setminus \{s,t\} : 
\begin{matrix}
  \sum_v x_{s,v} - \sum_u x_{u,s} \ge 1 \\
  \sum_u x_{u,t} - \sum_v x_{t,v} \ge 1 \\
  \sum_p x_{u,p} - \sum_q x_{q,u} \ge 1
\end{matrix}
\]

Note that the first condition ensures that the path originates from the source, the second condition ensures the path reaches the target, and the third condition ensures it's a path.

\paragraph{Example: Single machine scheduling of interval jobs}
Schedule jobs on a processor.
Each job can be scheduled in one of a finite collection of allowed time intervals.
Scheduling a job $j$ at interval $I$ imposes $w(I)$ load and $p(I)$ profit.
Find the most profitable subset of intervals, at most one interval per job, such that the total load at each time is at most 1.

Define the variables as $x_I$ for each possible interval I.

\subsection{Branch and Bound}
A.K.A. Dakin's Algorithm.
Let $Z^* = -\infty$ be the incumbent value, and start from the root.

Bound: Solve relaxed LP problem:
\begin{enumerate*}
  \item If infeasible, prune the branch.
  \item Let $U$ be the objective value. It is an upper bound on the possible values for this branch.
  \item If $U < Z^*$, then prune as the branch cannot produce the optimal solution.
  \item If all variables are integers, then $Z^* \gets U$.
\end{enumerate*}

Branch: Select a leaf with a non-integer variable $x^*$, branch into two sub-LPs: $x \le \lfloor x^* \rfloor$ and $\lceil x^* \rceil \le x$.

\paragraph{Examples}

\begin{itemize*}
  \item \href{http://optlab-server.sce.carleton.ca/POAnimations2007/MILP.html}{Interactive example}
  \item \href{http://optlab-server.sce.carleton.ca/POAnimations2007/BranchAndBound.html}{Scheduling example}
\end{itemize*}

\paragraph{Example: Weighted Vertex Cover}
Given a graph $G=(V,E)$ with non-negative weights $w(v)$ on the vertices.
Find a minimum-cost set of vertices $S$ such that all the edges are covered.
An edge is covered iff at least one of its endpoints is in $S$.

Reminder: Vertex cover is NP-Complete. The best known approximation factor is $2-\left( \frac{\log \log |V|}{2 \log |V|} \right)$

Let the variables be $x(v)$ for $v\in V$ as whether $v$ is in the cover or not.
Minimize $\sum_{v\in V} w(v) \times (v)$ such that $x(v) + x(u) \ge 1 \; \forall (u,v) \in E$ and $x(v) \in \{0,1\} \; \forall v\in V$.

\section{Next week}
More linear programming

\end{document}
