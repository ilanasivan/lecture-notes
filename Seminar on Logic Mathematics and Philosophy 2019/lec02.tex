\documentclass{idc_msc}

\title{Seminar on Logic Mathematics and Philosophy \\\large Lecture 02}
\date{2018-10-21 \\ Last edited \currenttime\ \today}
\author{Lecture by Dr. Udi Boker\\Typeset by Steven Karas}

\AtEndDocument{\bibliographystyle{plain}\bibliography{biblio}{}}

\begin{document}

\maketitle

\nocite{benacerraf1983philosophy}

\paragraph{Disclaimer}

These lecture notes are based on the seminars on Logic, Mathematics, and Philosophy; led by Dr. Udi Boker at IDC Herzliyah in the fall semester of 2018/2019.
Sections may be based on the lecture slides prepared by Dr. Udi Boker.

NOTE: I will try to upload a revised version of these notes with the missing parts filled in.

\section{Course Admin}

Nothing new, just a reminder regarding the 

\section{Formal Logics}

Formal logic deals with inference of formal concepts, and formal concepts only.

Assumptions are given, and listed.
For example:
\[ \alpha,\; \alpha \to \beta\]

We then have a claim, for example: \(\beta\).

From this, the proof follows:

\[\alpha, \; \alpha \to \beta \models \beta\]

A formal logic system is a combination of syntax, semantics, and optionally a proof system.
Ideally, the consequence relation of the proof system matches the semantics of the formal logic.

\clearpage
\section{First Order Logic}

Frege developed first order logic during the 19th century, and it was widely adopted by the 20th century.
First order logic is also called predicate calculus or predicate logic.
\marginpar{Steven: \href{https://teorth.github.io/QED/}{QED}}

\subsection{Syntax}

Variables are denoted as \(x_0,\; x_1,\; x_2,\; \ldots\).

Relations are denoted as \(\lnot,\; \lor,\; \land,\; \to\).

Quantifiers are \(\forall\) and \(\exists\), which must be followed by a variable.

% I missed the rest of the notation from slide 7?, and he didn't upload slides before the lecture started

Constants are also allowed, for example: \(0,\; \emptyset,\; \mathrm{David}\).

Functions are denoted with their arity: \(+,\; \lfloor \rfloor,\; \cap,\; \mathrm{Father}\).

Finally, relations are also allowed: \(\in,\; \subset,\; \mathrm{contains}\).

For first order logic, we allow nouns and expressions.
Nouns evaluate to an element of the universe, and formulas evaluate to some true/false value.
Functions operate on nouns, and evaluate to a noun.
Relations operate on nouns, and evaluate as a formula.

Within a formula, we refer to variables as free or bound.
Bound variables are those which are within a quantifier.
Note that a quantifier may only cover part of a formula, not all of it.
Variables within a noun are always free.

\paragraph{Goldbach Conjecture}

The Goldbach conjecture has gone unproven for several hundred years.
The conjecture states that every even number is the sum of two prime numbers.
Formally:

\[
\begin{aligned}
  \mathrm{Even}(x) \to& \exists y\; x = y + y \\
  \forall x \,\mathrm{Even}(x) \to& \exists y \exists z \, (x = y + z \land \mathrm{Prime}(y) \land \mathrm{Prime}(z)) \\
\end{aligned}
\]

A proposition is a formula that has no free variables.
Intuitively, a proposition always evaluates to either true or false.
Whereas a generic formula may depend on the values of the free variables.

\subsection{Axiomatic Theories}

A theory is defined as a collection of propositions that are called axioms, which we assume to evaluate to true.
It's widely considered as good if we can reduce the number of axioms we use for a theory.

Peano Arithmetic uses a dictionary of \(\Sigma=\{0, s, + , \cdot, <\}\) with ten axioms:

\begin{enumerate}
  \item There is no number smaller than 0: \(\forall x\, (\lnot (x < 0))\)
  \item There is no number whose successor is 0: \(\forall x\, (\lnot(s(x) = 0))\)
  \item The successor function is 1-1: \(\forall x \forall y\, \left(s(x)=s(y) \to (x=y)\right)\)
  \item 0 is neutral w.r.t. addition: \(\forall x \, (x + 0 = x)\)
  \item Multiplication by 0 is 0: \(\forall x \, (x \cdot 0 = 0)\)
  \item Addition by the successor: \(\forall x\forall y\, (x + s(y) = s(x + y))\)
  \item Multiplication by the successor
  \item Ordering by the successor
  \item Ordering is complete
  \item Induction: \((\alpha(0) \land \forall x\; (\alpha(x) \to \alpha\left(s(x)\right)) \to \forall x\; \alpha(x)\)
\end{enumerate}

% I missed a few of the formal axioms due to lack of time

\subsection{Structures}

For a logic system, we define a structure:

\begin{itemize}
  \item A nonempty set of elements called the domain \(D^M\)
  \item Constants are interpreted as a specific element \(c^M\)
  \item A n-ary function \(f\) is interpreted as a complete function \(f^M\) which takes \(n\) elements
  \item ...
  % yet again, didn't get them all
\end{itemize}

Propositions that evaluate to true for a given structure are said to be satisfied, or that the structure models them.

A structure models a theory if it satisfies all the axioms for a theory.

A theory is called a tautology if it holds for all structures.
A theory is called satisfiable if there is a structure for which it holds.
A theory is called a contradiction if it does not hold for all structures.

% many examples were given

\subsection{Semantic Consequence}

Let \(\alpha\) be a proposition and \(S\) be a set of propositions.
\(\alpha\) is semantically consequential, denoted as \(S \models \alpha\) if all models of \(S\) also model \(\alpha\).

Let \(\alpha\) and \(\beta\) be formulas. We say that they are semantically equivalent, denoted as \(\alpha \equiv \beta\) if \(\{\alpha\} \models \beta\) and \(\{\beta\} \models \alpha\).

Note that the following are tautologies:

\begin{itemize}
  \item \(\alpha \to \beta \equiv \lnot \alpha \lor \beta\)
  \item \(\lnot(\alpha \lor \beta) \equiv \lnot \alpha \land \lnot \beta\)
  \item ...
\end{itemize}

\subsection{Proof system}

A proof system is composed of logical axioms and inference rules.
If a proposition is inferred by syntax, we denote this as \(S \vdash \alpha\).
A proof system is called sound if for any claim and assumptions \(S \vdash \alpha\) implies \(S \models \alpha\).
A proof system is called complete if for any claim and assumptions \(S \models \alpha\) implies \(S \vdash \alpha\).

There are many different proof systems; we will see a Hilbert-style proof today.

\subsubsection{Hilbert-style proofs}

A formula \(\alpha\) is inferred from the set of formulas \(S\), denoted as \(S \vdash \alpha\) if there is a finite sequence of formulas, such that the sequence concludes with \(\alpha\) wherein each formula in this sequence is either a logical axiom, belongs to \(S\) as an assumption, or is derived from prior formulas using inference rules.

\subsection{G\"odel's Completeness Theorem}

G\"odel proved in his doctoral dissertation from 1929 that there exists a complete and sound proof system for First Order Logic.

% He showed an 8-axiom proof system with modus ponens

% He them showed a trivial proof using this proof system

\subsection{Consistency}

A set of propositions is consistent if it not possible to infer a contradiction from them.
This is a syntactic concept, whereas satisfaction is a semantic concept.
If a proof system is sound and complete, then a consistent set of propositions is a satisfiable theory.

For an unsatisfiable set of propositions \(S\), there are no structures that satisfy it.
As such, all the structures that satisfy \(S\) also satisfy \(\alpha\) and its converse \(\lnot \alpha\) in a degenerate way.
Therefore, \(S\) is inconsistent.

\subsection{Definability}

We say that a theory defines a specific structure if it is the only structure that satisfies all the propositions of that theory.

We call structures isomorphic if they are identical short of naming.
Semi-formally, we call structures isomorphic if there is a bijective (1-1 and onto) mapping between them that preserves the meaning of the structure.
This means that the mapping must preserve the types of the elements.

Formally, let \(\Sigma\) be the dictionary of the structures \(M_1\) and \(M_2\) with respective domains \(D_1\) and \(D_2\).
We say that \(M_1\) and \(M_2\) are isomorphic iff there is a mapping \(f : D_1 \to D_2\)...

% I missed the rest of the formal proof

Formally, we say that the structure \(M\) over the dictionary \(\Sigma\) is definable in first order logic if there is a theory \(T\) from first order logic over \(\Sigma\) such that \(M\) is the only structure up to isomorphism that satisfies \(T\).

\subsubsection{Compactness of FOL}

A set of formulas in FOL are satisfiable iff every finite subset is satisfiable.
As a restricted case of this, we will show this where the formulas are propositions, specifically that a theory \(T\) has a model iff...

% he's moving way too fast to keep track

Assume that \(T\) is not satisfiable and we will show that there is a finite subset that is not satisfiable.
Recall that we showed the equivalence of consistency and satisfiability.
Therefore, \(T\) is not consistent, which means there is a proof system that can conclude a contradiction.
This means that there exists some proposition for which \(T \vdash \alpha\) and \(T \vdash \lnot \alpha\).
Consider the finite set of propositions that appear in the Hilbert proof of \(T \vdash \alpha\).
Consider the propositions from this proof that are from \(T\) and denote them as \(T'\), which is a finite subset of \(T\).
Similarly, do the same for \(T \vdash \lnot \alpha\) and denote \(T''\).
\(T' \cup T''\) is finite, is a subset of \(T\), and is contradictory.

We will see in two weeks that Peano arithmetic does not define the natural numbers.
Worse, we will show that no First Order Logic can define the natural numbers.

\subsubsection{L\"owenheim–Skolem Theorem}

If a theory in First Order Logic has an infinite model (say, with an infinite domain), then it has a model for all infinite cardinalities.

\end{document}
