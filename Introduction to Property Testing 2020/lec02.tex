\documentclass{idc_msc}

\title{Introduction to Property Testing \\\large Lecture 2}
\date{2020-04-22 \\ Last edited \currenttime\ \today}
\author{Lecture by Dr. Reut Levi\\Typeset by Steven Karas}

\AtBeginDocument{\maketitle}
\AtEndDocument{\vfill\bibliographystyle{plain}\bibliography{../biblio}{}}

\begin{document}

\nocite{goldreich2017introduction}

\paragraph{Disclaimer}

These notes are based on the lectures for the course Introduction to Property Testing, taught by Dr. Reut Levi at IDC Herzliyah in the spring semester of 2019/2020.
Sections may be based on the lecture slides prepared by Dr. Reut Levi.

\section{Agenda}

  \begin{itemize}
    \item Recap
    \item Additional examples of property testing
  \end{itemize}

\section{Binary Sequences}

Note that in all the examples today we will be using binary sequences as our domain with relative Hamming distance as our distance metric.

Let \(x,z \in {\{0,1\}}^{*}\) be binary sequences.
Let \(\delta(x,z) = \begin{cases}|\{i \in [1x1] : x_i \ne z_i\}| & \textrm{if } |x| = |z| \\ \infty & \textrm{otherwise}\end{cases}\)

Let \(S\) be some set of binary sequences.
We define the distance of \(x\) from \(S\) as:

\[
  \delta_S(x) = \min_{z \in S} \{\delta(x,z)\}
\]

We define distinguishing sequences more distant than some \(\varepsilon\) for randomized algorithms as if \(x \in S\) then the algorithms accepts with probability \(> \frac{1}{2}\), and if \(x\) is more than \(\varepsilon\) far from \(S\) then rejects with probability \(> \frac{1}{2}\)\footnote{I use \(\frac{1}{2}\) instead of the \(\frac{2}{3}\) used in the slides}

\subsection{Majority}

We define the property of majority as the set of all sequences which have strictly more than half of their bits as 1.

\[
  \mathrm{MAJ} = \left\{x : \sum_{i = 1}^{|x|} x_i > \frac{|x|}{2}\right\}
\]

\subsubsection{Lower bound on number of queries}

Last time we proved this for any \(A\) that makes \(< \frac{n}{3}\) queries.
We claim that if \(X_n \sim D_1\) and \(Y_n \sim D_2\) then

\[
  \left| \Pr[A(X_n) = 1] - \Pr[A(Y_n) = 1] \right| \le \frac{q}{n}
\]

where A is an algorithm that makes \(q\) queries.

\paragraph{Proof}

As a sketch, we think of a random process that answers queries on the fly as such:

To determine \(X_n\)s, first pick \(i\) u.a.r.\footnote{uniformly at random} from \([n]\) as the special bit that we fix to be zero.
Then pick \(N\) u.a.r. from strings with Hamming weight \(\lfloor\frac{n}{2}\rfloor\) with a 0 in location \(i\).

To determine \(Y_n\)s, first pick \(i\) u.a.r. from \([n]\) as the special bit that we fix to be one.
Then pick \(Y\) u.a.r. from strings with Hamming weight \(\lfloor\frac{n}{2}\rfloor\) with a 0 in location \(i\) and then flip that bit.

The random process whenever queried on a "new" bit, it flips a coin to decide if this bit is special.
The other bits are chosen to be 1 or 0 on the fly as well.
Observe that if the random process does not pick a special bit then its answers are the same regardless of if it is a NO or YES instance process.

The probability to pick a special bit if we make \(q\) queries is therefore \(\frac{q}{n}\) from the union bound.

\subsubsection{Limitations of deterministic algorithms}

Any deterministic algorithm that distinguishes between \(\mathrm{MAJ}\) and \(0.5\)-far from \(\mathrm{MAJ}\) must make \(\frac{n}{2}\) queries.

\paragraph{Proof}

Assume there exists an algoirthm \(A\) that makes \(q < \frac{n}{2}\) queries. Consider the execution of \(A\) such that each query is answered with 0.
This execution is consistent with the all-0 string and a string with Hamming weight \(n - q\) ("1"s everywhere else).

\subsection{Symmetric properties}

We define a symmetric property as for any \(x \in {\{0, 1\}}^{*}\) and every permutation over \([|x|]\), \(x \in S\) iff \(X_{\Pi(1)}, \ldots, X_{\Pi(k)} \in S\) then \(S\) is symmetric.

For any symmetric property of binary sequences \(S\), there exists a randomized algorithm that makes \(O(\frac{1}{\varepsilon^2})\) queries and distinguishes between \(x \in S\) and \(x\) that are \(\varepsilon\) far from \(S\).
Note that we can guarantee this for any fixed alphabet, and this does generalize for an unbounded alphabet (exercise 1.3).

\paragraph{Proof}

Note that we can decide if \(x \in S\) or not just by knowing \(wt(x)\).
More formally:

\(\forall n \exists W_n \subseteq [n]\) such that \(\forall x \in {\{0, 1\}}^{n}\)

% I wasn't able to scribe the rest of the proof as i was distracted and she scrolled off...

\subsection{Beyond symmetric properties}

\subsubsection{Sorted}

A sequence \(x \in {\{0,1\}}^{*}\) is sorted iff \(x_i \le x_{i+1} \;\forall i \in [|x|-1]\).
\marginpar{In other words, the sequence is all 0s followed by all 1s (may be missing 0s or 1s)}

There exists a randomized \(O(\frac{1}{\varepsilon})\) time algorithm for \(\varepsilon\)-testing if a sequence is sorted.

\paragraph{Algorithm}

On input \(x \in {\{0,1\}}^{n}\), query \(x\) in locations \(\frac{\varepsilon n}{2} \cdot i\) for every \(i \in [\frac{2}{\varepsilon}]\).
Then query \(x\) at \(m = O(\frac{1}{\varepsilon})\) locations u.a.r.
Accept iff the induced substring is sorted.

\paragraph{Correctness}

If substring induced by step 1 is not sorted, then we reject.
If it is sorted and \(x \in S\) then we can determine \(x\) up to a single interval; specifically the interval where we switch from 0s to 1s.
If \(x\) is \(\varepsilon\)-far from SORTED, then there exists \(\ge \frac{\varepsilon n}{2}\) indexes such that we can reject if we sample those indexes.

% she keeps scrolling the proofs very quickly, and I don't have time to scribe the whole proof.
% according to her, the proof follows very similarly to the proof for error correction codes


\pagebreak
\section{Formal definitions}

\begin{itemize}
  \item Properties of function which represent obejcts
  \item prop e by querying the function
  \item n - size of domain, w.l.o.g. domain \([n]\)
  \item Range \(R_n\) which may depend on \(n\)
  \item property: set \(\Pi_n\) of functions from \([n]\) to \(R_n\)
  \item For the sake of asymptotic analysis and algorithmic uniformity we let \(n\) vary and consider testers for \(\Pi = \bigcup_{n \in \mathbb{N}} \Pi_n\).
  \marginpar{\vspace{-6\baselineskip}for example, this means that some algorithms will branch if \(n\) is even or odd}
\end{itemize}

Input is oracle access to \(f : [n] \to R_n\) with proximity parameters \(\varepsilon > 0\) and \(n\).

Distance is defined for two functions \(f, g : [n] \to R_n\).

% I missed some of the definitions while looking up how to float a marginpar up in latex

\subsection{Property Tester}

A tester for property \(\Pi\) is a probabilitistic algorithm denoted \(T\) that receives as input query access to \(f: [n] \to R_n\) and parameters \(n\) and \(\varepsilon\) such that:

\begin{enumerate}
  \item if \(f \in \Pi_n\) then \(\Pr[T(f, n ,\varepsilon) = 1] \ge \frac{2}{3}\)
  \item if \(\delta_\Pi(f) > \varepsilon\) then \(\Pr[T(f, n, \varepsilon) = 0] \ge \frac{2}{3}\)
\end{enumerate}

If \(\Pr[T(f, n, \varepsilon) = 1] = 1\) for every \(f \in \Pi_n\) then \(T\) has one sided error.
Otherwise, we say \(T\) has two sided error.

A tester has query complexity \(q : \mathbb{N} \times (0, 1] \to \mathbb{N}\) if on input \(n, \varepsilon\) and oracle access to any \(f : [n] \to R_n\) the tester makes at most \(q(n, \varepsilon)\) queries.
Note that in this field we want algorithms that are sublinear to \(n\).

\subsection{Nonadaptivity}

The queries of an algorithm are determined by \(n, \varepsilon\), not the results of the query.

\section{Homework}

She suggested writing extremely formal and rigorous solutions.
From the book she wants us to do exercise 1.1 and exercise 1.3 due two weeks from now (this will be added in moodle)

\section{Note regarding missing sections in these notes}

The lecture notes were scrolled a little too fast for me to scribe them in real time, so I will take the effort to update these notes in a few days.

\section{For next time}

  Due to independence day, the next lecture will take place in two weeks.

\end{document}
