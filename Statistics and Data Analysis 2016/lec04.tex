\documentclass[a4paper]{article}

\usepackage[T5]{fontenc}
\usepackage[utf8]{inputenc}
\usepackage{amsfonts}
\usepackage{mathtools}
\usepackage[iso]{datetime}
\usepackage{tabu}
\usepackage[colorlinks=true,urlcolor=blue,linkcolor=black]{hyperref}

\title{Statistics and Data Analysis\\\large Lecture 4}
\date{2016-11-29 \\ Last edited \currenttime\ \today}
\author{Lecture by Dr. Zohar Yakhini\\Typeset by Steven Karas}

\newenvironment{itemize*}%
  {\begin{itemize}%
    \setlength{\itemsep}{0pt}%
    \setlength{\parsep}{0pt}%
    \setlength{\parskip}{0pt}}%
  {\end{itemize}}

\newenvironment{enumerate*}%
  {\begin{enumerate}%
    \setlength{\itemsep}{0.5pt}%
    \setlength{\parsep}{0pt}%
    \setlength{\parskip}{0pt}}%
  {\end{enumerate}}

\begin{document}

\maketitle

\section{Normal Distribution}
Sometimes called the Gaussian distribution. Colloquially referred to as a bell curve.

\[f(x)=\frac{1}{\sigma\sqrt{2\pi}}\, e^{\frac{-(x-\mu)^2}{2\sigma^2}}\]

\subsection{Standard Normal}
A special case of the Normal distribution with $\mu=0$ and $\sigma=1$.

\subsection{CDF}
The function is not integrable. Lookup tables and numeric computation are used instead.

\paragraph{Example: Bicycle Oil}
In a bicycle shop, they sell oil. When the stock of oil drops below 20, they order new stock. However, the delivery takes a full day to arrive. What are the chances of running out of stock while waiting for the delivery?

Let the daily demand be $\sim N(\mu=15, \sigma=6)$ bottles of oil.

\subsection{Z-Shift}

\[Z=\frac{x-\mu}{\sigma}\]
\[\frac{dz}{dx}=\frac{1}{\sigma}\]
\[\frac{1}{2\sqrt{\pi}} \int_I e^{\frac{(x-\mu)^2}{2\sigma^2}} dx\]
\[=\frac{1}{2\sqrt{\pi}} \int_I e^{\frac{z^2}{2}} dz\]

Using this technique, we can transform any integral on any normal distribution to the standard normal.

\paragraph{Example: Bicycle Oil}

\[z=\frac{(x-\mu)}{\sigma}=\frac{20-15}{6}=0.83\]

From our reference tables/tools, we get that this is $0.2033$

\subsection{Sampling from a Uniform distribution to any distribution}

Let $f$ be a CDF. $f(x)=y$, and because $f$ is monotonous, $f^{-1}(y)=x$.

Therefore, $P(f^{-1}(\text{RAND})<x)=P(\text{RAND}<y)=y=f(x)$

In short, to generate, we draw uniform random numbers and use the inverse CDF to generate the values according to the distribution.

\subsection{}

\end{document}
