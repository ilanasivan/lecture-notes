\documentclass[a4paper]{article}

\usepackage[T5]{fontenc}
\usepackage[utf8]{inputenc}
\usepackage{amsfonts}
\usepackage{mathtools}
\usepackage[iso]{datetime}
\usepackage{tabu}
\usepackage[colorlinks=true,urlcolor=blue,linkcolor=black]{hyperref}

\title{Statistics and Data Analysis\\\large Lecture 8}
\date{2016-12-27 \\ Last edited \currenttime\ \today}
\author{Lecture by Dr. Zohar Yakhini\\Typeset by Steven Karas}

\newenvironment{itemize*}%
  {\begin{itemize}%
    \setlength{\itemsep}{0pt}%
    \setlength{\parsep}{0pt}%
    \setlength{\parskip}{0pt}}%
  {\end{itemize}}

\newenvironment{enumerate*}%
  {\begin{enumerate}%
    \setlength{\itemsep}{0.5pt}%
    \setlength{\parsep}{0pt}%
    \setlength{\parskip}{0pt}}%
  {\end{enumerate}}

\newcommand{\mean}[1]{\mkern 1.5mu\overline{\mkern-1.5mu#1\mkern-1.5mu}\mkern 1.5mu}

\begin{document}

\maketitle

\section{Review: Correlations}

Mathematical measure of the strength of a relationship between two variables.
We discussed assessing statistical significance.
A relationship can be strong, but not significant.
A relationship can be weak, yet significant.
The key factor is the size of the sample.
Small samples can produce strong correlation by chance.
Large samples can easily show significance, but we need to evaluate this together with relationship strength.

\section{Spearman Correlation - Derivation}
Convert the $u_i$s to $x_i$s, and the $r_i$s to $y_i$s (ranked).

$Sp(u,r)=Pearson(x,y)$

\[
\frac{\sum(x_i-\mean{x})(y_i-\mean{y})}
{\left(\sum(x_i-\mean{x})\sum(y_i-\mean{y})\right)^{1/2}}
\]

\[
Sp(\rho)=1-\frac{6\sum d_i^2}{n(n^2-1)}
\]
where $d_i=x_i-y_i$

\[
\frac{1}{2} \sum d_i^2 = \frac{1}{2} \sum (x_i - y_i)^2 = \frac{1}{2} \left(\sum x_i^2 -2\sum x_i y_i + \sum y_i^2 \right)
\]
\[=\sum x_i^2 - \sum x_i y_i\]

\[
\frac{\sum x_i y_i}{\sum x_i^2}
= 1 - \frac{\frac{1}{2} \sum d_i^2}{\sum x_i^2}
\]

\[x_i \to a_i=x_i-\frac{n+1}{2}\]
\[y_i \to b_i=y_i-\frac{n+1}{2}\]

Proof of this by induction is left as an exercise:
\[\sum_{i=1}^k i^2=\frac{k(k+1)(2k+1)}{6}\]

\[\sum_{i=-\frac{n-1}{2}}^{\frac{n-1}{2}} = \sum_{i=1}^{\frac{n-1}{2}} 2i^2 = 2 \cdot \frac{1}{6} \frac{n-1}{2} \frac{n+1}{2}\cdot n\]

\[=\frac{1}{12}n(n^2-1)\]

\section{Convolution of two distributions}
An example of the function ConvDbns and it's implementation in matlab was given on the board.

\[c = conv(u,r)\]
\[c_i=\sum_{j} u_j r_{i-j}\]
\[P(X+Y=1)=\sum_j P(X=j)P(Y=i-j)\]

\subsection{N-Fold convolution}
trivial to implement using a generic convolution.

\section{Hypergeometric distribution}
Given a class of 40 students, of whom 10 got an A on the exam. The last names of 8 of them begin with a letter in the range A-M.

\subsection{Example: seating}
$N = 40\text{ seats}$
$B = 30\text{ students}$
$n = 20\text{ students on right half of class}$
$b \ge 20\text{ sitting on the right}$

\[
\frac{X\sim Binom(40, 1/2)}{P(X=30)}
\]

\[
HG(N,B,n,b) = \frac{\binom{B}{b}\binom{N-B}{n-b}}{\binom{N}{n}}
\]

\subsection{Example: cold treatment}

\begin{tabu} to \linewidth{|r|c|c|c|}
\hline
& yes & no & Total \\
\hline
1-3 days & 86 & 19 & 105 \\
\hline
4-7 days & 16 & 79 & 95 \\
\hline
Total & 102 & 98 & 200 \\
\hline
\end{tabu}

We want to evaluate this treatment via the HG null model:

$N=200$
$B=105\text{ people whose cold lasted $<3$ days}$
$n=102\text{ people who took the treatment}$
$b=86$

\[HGT(N, B, n, b) = \sum_{t=b}^{\min(n, b)} HG(N,B,n,b)\]

For this example, the p value is $10^{-15}$

\section{Contingency tables with finer partitions}

\subsection{Example: Lotto}

6 of 42 numbered balls are drawn at random without replacement.
You wrote 6 on your card ahead of time.
How many match?

\[HG(42,6,6,0) = \frac{\binom{6}{0}\binom{36}{6}}{\binom{42}{6}} \approx 0.37\]
\[HG(42,6,6,1) \approx 0.43\]
\[HG(42,6,6,5) \approx 0.00004\]
\[HG(42,6,6,6) \approx 0.00000002\]
\[HG(42,7,6,6) = 7 \cdot HG(42,6,6,6)\]

\section{Multi-Hypergeometric}
$k$ different types of items
total of $m$ items
$m_i$ items of type $i$
We select $n$ out of $m$ with all possible such selections being equiprobable

\paragraph{Example: shapes}

3 circles, 5 squares, 9 triangles, choosing 3 items:

$m_1=3 \quad m_2=5 \quad m_3=9 \quad k=3$

% TODO: copy generic form from slide 25.
\[P(Y_i=j_i \;|\; \forall i \in [1,k])\]

\section{Confidence intervals}

Reminder: Wald confidence interval:
\[ p^* \pm \Phi^{-1} (0.95)\sqrt{p^* \cdot q^* / n} \]

\paragraph{Example: cold treatment continued}
The confidence interval is around $0.055$, but we observed $0.64$. Therefore, we can reject this with a p-value of virtually 0:

% TODO: fill in from slide.

\section{Comparing two samples with student's t-test}
Matlab function ttest2.

\section{HW3}

\subsection{Haberman data}
Data from chicago hospitals for breast cancer surgeries.

Age, year of operation, Number of positive axillary nodes detected, Survival status (1=survived >5 years, 2=died within 5 years)

Do patients with more than 10 positive nodes have lesser survival?



\end{document}
