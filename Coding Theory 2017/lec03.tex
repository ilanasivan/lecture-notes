\documentclass{idc_msc}

\title{Coding Theory\\\large Lecture 3}
\date{2017-11-16 \\ Last edited \currenttime\ \today}
\author{Lecture by Dr. Elette Boyle\\Typeset by Steven Karas}

\newcommand{\Fq}[1][q]{{\mathbb{F}_{#1}}}
\DeclareMathOperator*{\RS}{RS}

\begin{document}

\maketitle

\paragraph{Disclaimer}

These lecture notes are based on the lecture for the course Coding Theory, taught by Dr. Elette Boyle at IDC Herzliyah in the fall semester of 2017/2018.
Sections may be based on the lecture slides written by Dr. Elette Boyle.

\paragraph{Admin}

\paragraph{Short Review}

\begin{itemize}
  \item \(n,k,d,\;\Sigma,q,\;\Delta\)
  \item Finite fields \(\Fq \quad (q=p^s)\)
  \item Linear codes = linear subspaces in \(\Fq^n\).
    Note: \([n,k,d]_q\) code vs \((n,k,d)_q\) code.
  \item Nice properties of linear codes (generator matrix \(G \in \Fq^{k \times n}\), parity check matrix \(H \in \Fq^{n-k \times n}\))
\end{itemize}

\paragraph{Agenda}

\begin{itemize}
  \item Singleton Bound\footnote{This is covered in chapter 4 of the book}
  \item Reed-Solomon Codes
  \item Maximum Distance Separable (MDS) codes
  \item Finite fields beyond \(\Fq[p]\; \text{p is prime}\)
\end{itemize}

\section{Singleton Bound}

Basically, this is the loose bound that you cannot provide a code better than the total size of the code space divided by the size of the Hamming ball.

\paragraph{Theorem}

For every \((n,k,d)_q\) code, \(k \le n - d + 1\).\marginpar{This means any code, not just linear codes}

\paragraph{Proof}

Let \(C\) be an \((n,k,d)_q\) code.
Let \(\vec{c}_1,\ldots,\vec{c}_M\) be all the codewords in \(C\), where \(k=\log_q M\).

We want to show that if \(C\) has distance \(d\), then \(M \le q^{n-d+1}\).

For each \(i \in [M]\), mark \(\vec{c'}_i\) as the prefix of \(\vec{c}_i\) of length \(n - d + 1\).

We want to show that \(\forall i \ne j \in [M] : \vec{c'}_i \ne \vec{c'}_j\).
Suppose the contradiction. Thus, \(\exists \vec{c'}_i = \vec{c'}_j\).
This implies that \(\Delta(\vec{c}_i, \vec{c}_j) \le d-1\) because the suffixes are of size \(d-1\).
This contradicts that \(d\) is the distance of \(C\).


\clearpage
\section{Reed-Solomon Codes}

The intuition is that we can map messages into polynomials and encode extra interpolation points.
This allows us to recover from erasures. In future lectures, we will introduce efficient algorithms for recovering from general errors.

Basically, oversampling low degree polynomials.

\subsection{Polynomials}

For a finite field \(\Fq\), a function \(F(x) = \sum_{i=0}^\infty f_i x^i\) with \(f_i \in \Fq\) is called a polynomial over \(\Fq\).

For \(F(x) = \sum_{i=0}^d f_i x^i \quad (f_d) \ne 0\) the degree of \(f\) is \(d\), marked as \(\deg(F) = d\).

\(\alpha \in \Fq\) is a root of \(F(x)\) if \(F(\alpha) = 0\).

Note that adding and multiplying polynomials is as usual, with coefficient arithmetic over \(\Fq\).

\paragraph{Bound on the possible roots of polynomials}

A nonzero polynomial \(F(x)\) of degree \(t\) over \(\Fq\) can have at most \(t\) roots in \(\Fq\).

\paragraph{Proof}

Proof by induction on \(t\).
If \(t = 0\), it is trivially true.

Say that \(F(x)\) is a nonzero polynomial of degree \(t \ge 1\).
If \(F(x)\) has no roots, then done.
Otherwise, let \(\alpha \in \Fq\) be a root of \(F\).

We claim that \(F(x) = (x - \alpha)g(x)\) where \(\deg(g) = \deg(f) - 1\).

It always holds that \(F(x) = (x- \alpha)g(x) + R(x)\) where \(\deg(g) \le \deg(f) - 1\), and \(\deg(R) \le 1 - 1\)\footnote{full proof elided for brevity}.
Therefore, \(\deg{R} = 0\), which means that \(R\) is constant.
Because \(\alpha\) is a root, \(F(\alpha) = 0\), and therefore:

\[
\begin{aligned}
  \underbrace{(\alpha - \alpha)}_{=0}g(\alpha) + R(\alpha) = 0 \\
  R(\alpha) = 0 \\
  R \equiv 0
\end{aligned}
\]

By inductive hypothesis, \(g\) has at most \(\deg(g) \le \deg(f) - 1\) roots.
The roots of \(f\) are the roots of \(g\) and \(\alpha\).
Thus, the number of roots of \(f \le \deg(f)\).

\subsection{Formally}

Let \(\Fq\) be a finite field.
Let \(\alpha_1,\ldots,\alpha_n \in \Fq\) be distinct elements (sometimes called the evaluation points).
Define an encoding procedure \(\RS : \Fq^k \to \Fq^n\) for a chosen \(k \le n \le q\):

\begin{enumerate}
  \item An input message \(\vec{m} = (m_0,\ldots,m_{k-1}) \in \Fq^k\).
    This message is mapped to a degree \(\le k - 1\) polynomial where \(F_{\vec{m}}(x) = \sum_{i=0}^{k-1} m_i x^i\).
  \item The output \(\RS(\vec{m})\) is the evaluation of \(f_{\vec{m}}(x)\) on all \(n\) evaluation points \(\alpha_i\):
    \(\RS(\vec{m}) = (F_{\vec{m}}(\alpha_1),\ldots,F_{\vec{m}}(\alpha_n))\).
    The image of this mapping is the RS code.
\end{enumerate}

A common choice is \(n = q-1\) and \(\alpha = \Fq \setminus \{0\}\).

\subsection{Proof of linearity}

We know for polynomials that \(\vec{m},\vec{m'} \in \Fq^k\), it holds that:

\[
\begin{aligned}
F_{\vec{m}}(x) + F_{\vec{m'}}(x)
& = \sum_{i=0}^{k-1} m_i x^i + \sum_{i=0}^{k-1} m'_i x^i \\
& = \sum_{i=0}^{k-1} (m_i + m'_i) x^i \\
& = f_{\vec{m} + \vec{m'}}(x)
\end{aligned}
\]

For any \(a \in \Fq\), \(aF_{\vec{m}}(x) = F_{a \vec{m}}(x)\).

This implies:

\[
\begin{aligned}
\RS(\vec{m}) + \RS(\vec{m'})
& = (F_{\vec{m}}(\alpha_1),\ldots) + (\ldots) \\
& = (F_{\vec{m}} + F_{\vec{m'}}(\alpha_1),\ldots) \\
& = (F_{\vec{m} + \vec{m'}}(\alpha_1),\ldots) + (\ldots) \\
& = \RS(\vec{m} + \vec{m'})
\end{aligned}
\]

Recall that the code is the image of \(\RS\) on all \(\Fq^k\) input so \(\RS(\vec{m} + \vec{m'})\) is a codeword.

Similarly, \(a \cdot \RS(\vec{m}) = \ldots = \RS(a \cdot \vec{m})\) is also a codeword.

\subsection{Generator matrix}

Note that \(\vec{m} G = \RS(\vec{m})\).

This implies that some element of the codeword is \(F_{\vec{m}}(\alpha_j) = \sum_{i=0}^{k-1} \vec{m} \alpha_j^i\).
Therefore, we can define the elements of \(G\) as \(G_{ij} = \alpha_i^j\).

We call this matrix the Vandermonde matrix, which is a full rank\footnote{A formal proof is by expressing the determinant as a polynomial with each of the elements as roots}.

\subsection{Equivalence to Singleton Bound}

A Reed-Solomon code for parameters \(n,k,q\) is an \([n,k,(n-k+1)]_q\) code.
That is, RS matches the singleton bound \(k \le n-d+1\).

\paragraph{Proof}

Fix \(\vec{m} \ne \vec{m'} \in \Fq^k\).
Note that \(F_{\vec{m}},\, F_{\vec{m'}}\) are distinct, because they differ in at least 1 coefficient.
Both of these polynomials are of degree \(\le k-1\).

\[
\begin{aligned}
  \Delta(\RS(\vec{m}), \RS(\vec{m'}))
  & = \left|\{ \alpha_i \mid F_{\vec{m}}(\alpha_i) \ne F_{\vec{m'}}(\alpha_i)\}\right| \\
  & = \left|\{ \alpha_i \mid F_{\vec{m} - \vec{m'}}(\alpha_i) \ne 0 \}\right|
\end{aligned}
\]

We know that \(F_{\vec{m} - \vec{m'}}\) is not the zero polynomial.
We also know that \(F_{\vec{m} - \vec{m'}}\) has degree \(\le k\), because \(F_{\vec{m} - \vec{m'}} = F_{\vec{m}} - F_{\vec{m'}}\).
From earlier, we know that the number of roots of \(F_{\vec{m} - \vec{m'}}\) is \(\le (k-1)\).

\[
\begin{aligned}
\left|\{ \alpha_i \mid F_{\vec{m} - \vec{m'}}(\alpha_i) \ne 0 \}\right|
& = n - \left|\{ \alpha_i \mid F_{\vec{m} - \vec{m'}}(\alpha_i) = 0 \}\right| \\
& \ge n - \text{the number of zeros of \(F_{\vec{m} - \vec{m'}}\)} \\
& \ge n - (k - 1)
\end{aligned}
\]

\subsection{Erasure decoding}

Simple interpolation works, but we won't cover it at the moment.

\subsection{Examples}

Consider the finite case \(F(x) = 2x^3 + x^2 + 5x + 6\) as a polynomial over \(\mathbb{F}_7\).

\[\deg(2x^3 + x^2 + 5x + 6) = 3\]

Consider codewords of a \([\overbrace{3}^n,\overbrace{2}^k]_3\) over \(\mathbb{F}_3\).
There should be \(q^k = 3^2 = 9\) such codewords.
The codewords should be in \(\Fq^n = \mathbb{F}_3^3\).
We'll just enumerate by brute force from the input space: \(\RS : \mathbb{F}_3^2 \to \mathbb{F}_3^3\)

\[
\begin{matrix}
\underbrace{(000)}_{\vec{m} = 00; F(\vec{m}) = 0} & \underbrace{(111)}_{\vec{m} = 01; F(\vec{m}) = 1} & \underbrace{(222)}_{\vec{m} = 02; F(\vec{m}) = 2} \\
\underbrace{(210)}_{\vec{m} = 10; F(\vec{m}) = x} & \underbrace{(021)}_{\vec{m} = 11; F(\vec{m}) = x+1} & \underbrace{(102)}_{\vec{m} = 12; F(\vec{m}) = x+2} \\
\underbrace{(120)}_{\vec{m} = 20; F(\vec{m}) = 2x} & \underbrace{(201)}_{\vec{m} = 21; F(\vec{m}) = 2x+1} & \underbrace{(012)}_{\vec{m} = 22; F(\vec{m}) = 2x+2}
\end{matrix}
\]


\clearpage
\section{Maximum Distance Separable codes}

A Maximum Distance Separable code is an \((n,k,q)_q\) code for which \(d = n - k + 1\) (i.e. it meets the singleton bound).
We've just shown that Reed-Solomon codes are MDS codes.

\subsection{Finite fields beyond \texorpdfstring{\(\Fq[p]\)}{Fp}}

Recall that there is exactly one finite field with a given cardinality, within isomorphism.

Recall polynomials over \(\Fq\)

A polynomial \(F(x)\) over \(\Fq\) is irreducible if \(\forall G_1(x), G_2(x)\) for which \(F(x) = G_1(x)G_2(x)\), it holds that \(\min(\deg G_1, \deg G_2) = 0\) (i.e. one of them must be a constant).

\paragraph{Theorem}

Let \(E(x)\) be an irreducible polynomial of degree \(s \ge 2\) over \(\Fq[p]\), where \(p\) is prime.
Then the set of polynomials \(\Fq[p][x]\) modulo \(E(x)\) denoted \(\frac{\Fq[p]}{(E(x))}\) is a field.

\paragraph{Proof Sketch}

The proof largely follows the proof that \(\Fq[p] = \frac{\mathbb{Z}}{(p)}\) is a field.

Note that the elements of \(\frac{\Fq[p]}{(E(x))}\) are all polynomials of degree \(\le s - 1\).

Recall that arithmetic over this field is \(\mod E(x)\), which means that for any \(G \mod E\):
\(G(x) = A(x)E(x) + R(x)\).

The additive identity is the zero polynomial.

The additive inverse \(F(x) = \sum_{i=0}^{s-1} f_i x^i\) is \(\sum_{i=0}^{s-1} (-f_i) x^i\).

The multiplicative identity is \(1\).

The multiplicative inverse can be shown to exist uniquely similar to how we proved for arithmetic \(\mod p\) where \(p\) is prime.

The number of elements is the number of distinct \(\deg \le s-1\) polynomials, which is \(p^s\), because the coefficients are in \(\Fq[p]\).

\paragraph{Existence of irreducible polynomials}

For any \(s \ge 2\) and \(\Fq[p]\) prime, there exists an irreducible polynomial over \(\Fq[p]\) of degree \(s\).
In fact, the number of such irreducible polynomials is \(\Theta\left(\frac{p^2}{s}\right)\).

Note that given a polynomial \(F(x)\) of degree \(s\), then we can efficiently\footnote{The proof is interesting, and it is recommended to read up on it later.} determine if \(F\) is irreducible.
We can do this by checking \(\gcd(F(x), x^{q^s}-x) \underrel{?}{=} F(x)\) since we knwo that \(x^{q^s}-x = \prod_{F\text{ irreducible}; \deg F = s} F\).

Because we can do this efficiently, we can start from \(\Fq[p]\), and find an irreducible polynomial to construct \(\mathbb{F}_{p^s}\).

Recall that when we constructed the Reed Solomon codes, we defined them over a finite field, and if we consider our encoding field as acting on polynomials in \(F_{p^s}\), we are effectively defining a field of polynomials over polynomials. However, while addition over the coefficient representation works naively, multiplication is non-trivial.

\subsection{Examples}

\(1 + x^2\) over \(\mathbb{F}_2\) is not irreducible, because:
\[
\begin{aligned}
1 + x^2
& = (1 + x)(1 + x) \\
& = 1 + 2x + x^2 \\
& = 1 + x^2
\end{aligned}
\]

\(1 + x + x^2\) is irreducible over \(\mathbb{F}_2\).

In order to have a nontrivial factorization over \(\mathbb{F}\), either \(\underbrace{x}_{a = 0 \text{ is a root}} \mid 1 + x + x^2\) and/or \(\underbrace{(1+x)}_{1 \text{ is a root}} \mid 1 + x + x^2\).
Neither is implied, and therefore \(1 + x + x^2\) is irreducible.

Note that having no roots is not equivalent to being irreducible over finite fields:

\((1 + x + x^2)^2\) is clearly not irreducible, yet has no roots.

Suppose that we take \(E(x) = 1 + x + x^2\), which is irreducible over \(\mathbb{F}_2\).
Taking \(\mod E(x)\) is saying that \(1 + x + x^2 = 0 \mod E(x)\), and \(x^2 = -(1 + x)\).

For example, take \(G(x) = x^5 + x^3 + x^2 + 1\).
We can substitute the \(x^2\) terms to reduce the degree and it's all arithmetic from there.

\section{Next week}

I will be missing the lecture next week because of Thanksgiving.

\end{document}
