\documentclass{idc_msc}

\usepackage{tikz}
\usetikzlibrary{graphs,positioning,quotes}

\title{Coding Theory\\\large Lecture 10}
\date{2018-01-02 \\ Last edited \currenttime\ \today}
\author{Lecture by Dr. Elette Boyle\\Typeset by Steven Karas}

\newcommand{\Fq}[1][q]{{\mathbb{F}_{#1}}}
\DeclareMathOperator*{\Vol}{Vol}
\DeclareMathOperator*{\RS}{RS}
\DeclareMathOperator*{\Pcheck}{Pcheck}
\DeclareMathOperator*{\Dec}{Dec}
\newcommand{\finiteuniform}[1][\mathbb{D}]{{\mathcal{U}_{#1}}}
\newcommand{\BSC}[1][p]{{\text{BSC}_{#1}}}

% \AtEndDocument{\bibliographystyle{plain}\bibliography{biblio}{}}

\begin{document}

\maketitle

\paragraph{Disclaimer}

These lecture notes are based on the lecture for the course Coding Theory, taught by Dr. Elette Boyle at IDC Herzliyah in the fall semester of 2017/2018.
Sections may be based on the lecture notes written by Dr. Elette Boyle.

\paragraph{Agenda}

\begin{itemize}
  \item Locally Decodable Codes (self-synchronizing)
  \item Hadamard Code
  \item Reed-Muller Codes
  \item Survey of known codes
  \item Cryptographic applications
\end{itemize}

\section{Review}

Last time we finished covering decoding binary expander codes.
Expander codes are a subset of Low Density Parity Check (LDPC).

Note that we can also consider more generic constructions of codes from bipartite graphs.
Each parity vertex \(r \in R\) gives a more general constraint on its neighbors.
Given a total ordering on the vertices, we can define a simple single-parity check code by defining the neighbors of \(r\) as a 

\section{Locally Decodable Codes (LDC)}

\paragraph{Intuition}

Suppose that we start with a very long data stream \(m\).
Instead of recovering the entire data stream, we simply want to robustly recover some subset of symbols.

A simple way to do this is to decode the entire codeword, look up the symbols we want, and then throw away the rest.
But we want to recover these symbols in sublinear time.

We could encode the data stream into blocks, which is good, but very poor error correction guarantees.

Locally Correctable Codes (LCC) are similar, but the goal is to recover symbols from the codeword itself.

\subsection{Formally}

A \(q\)-ary code \(C : [q]^k \to [q]^n\) is a \((r, \delta, \varepsilon)\) locally decodable code if there exists a randomized decoding algorithm \(\Dec : [q]^n \times [k] \to [q]\) that is robust and local.

Robustness is defined as if for any index \(i \in [k]\) we are able to recover from some received message \(\vec{y} \in [q]^n\) within \(\Delta(C(\vec{x}),\vec{y}) \le \delta n\) distance of the original message \(\vec{x} \in [q]^k\) the symbol \(x_i\) with probability \(\ge 1 - \varepsilon\):

\[\Pr[\Dec(\vec{y}, i) = x_i] \le 1 - \varepsilon\]

Locality is defined as reading at most \(r\) symbols of \(\vec{y}\).

% Sublinear complexity limits us to sublinear \(r\), and defining sublinear \(r\) is useful for cryptography

\paragraph{A note on the parameters}

Generally speaking, we want \(r\) to be small, \(\delta\) to be big, and \(\varepsilon\) to be small.

For data transmission, we consider \(\delta\) to be constant (e.g. \(\sim 1/4\)), but we want \(n\) to be small, and we don't care so much about \(r\), so long as \(r \le k\).

For cryptographic applications, we don't care much about \(\delta\), so long as \(\delta > 0\) and \(\varepsilon < 1/2\) is held as constant.
Mostly cares about the tradeoff between \(r\) and \(n\).

\subsection{Hadamard Code}

Recall that \(C_{\text{Had}} : \{0,1\}^k \to \{0,1\}^{2^k}\).

\[\vec{m} \to (\vec{m}, \vec{v})_{\vec{v} \in \{0,1\}^k}\]

Indexed by \(\vec{y} \in \{0,1\}^{2^k}\) by vectors \(\vec{v} \in \{0,1\}^k\).

% k times k by 2^k generator matrix with columns v gives us the codeword

We can recover \(m_i\) by considering both \(\vec{v}\) and \(\vec{v} \oplus \vec{e_i}\) where \(\vec{e_i}\) is the unit vector with position \(i\):

\[
\vec{m} \cdot (\vec{v} \oplus \vec{e_i})
= \vec{m} \cdot \vec{v} \oplus \vec{m} \cdot \vec{e_i}
= (\vec{m} \cdot \vec{v}) \oplus m_i
\]

\[m_i = \vec{m} \cdot \vec{v} \oplus \vec{m} \cdot (\vec{v} \oplus \vec{e_i})\]

\paragraph{Proof}

The \([\overbrace{2^k}^{n},\overbrace{k}^{k}]\) Hadamard Code is \((r=2, \delta, 2\delta)\) locally decodable.

Consider the following randomized decoding algorithm:

\begin{lstlisting}[frame=L,mathescape=true,title={Local decoding Hadamard}]
Dec($\vec{y}$, $i$):
  Sample a random vector $\vec{v} \gets \{0, 1\}^k$
  $y_{\vec{v}}$ = value of $\vec{y}$ at position $\vec{v}$
  $y_{\vec{v}\oplus \vec{e_i}}$ = value of $\vec{y}$ at position $\vec{v} \oplus \vec{e_i}$
  return $(y_{\vec{y}}) \oplus (y_{\vec{v} \oplus \vec{e_i}}) \in \{0,1\}$
\end{lstlisting}

Note that if the \(\vec{v}\) and \(\vec{v} \oplus \vec{e_i}\) positions of \(\vec{y}\) do not have error, then:

\[
\begin{aligned}
  (y_{\vec{v}}) \oplus (y_{\vec{v} \oplus \vec{e_i}})
  &= (\vec{m} \cdot \vec{v}) \oplus (\vec{m} \cdot (\vec{v} \oplus \vec{e_i})) \\
  &= \vec{m} \cdot \vec{v} \oplus \vec{m} \cdot \vec{v} \oplus \vec{m} \cdot \vec{e_i} \\
  &= m_i
\end{aligned}
\]

So \(\Pr_{\vec{v}}[\Dec(\vec{y}, i) = m_i] \ge \Pr_{\vec{v}}[\vec{v},\vec{v} \oplus \vec{e_i}\text{ not in error}]\).

Let \(S \subseteq \{0,1\}^n\) be the error positions in \(\vec{y}\).
Suppose that \(|S| \le \delta n\).

\[\Pr_{\vec{v}}[\vec{v} \in S] = \frac{|S|}{n} \le \delta\]
\[\Pr_{\vec{v}}[\vec{v} \oplus \vec{e_i} \in S] = \frac{|S|}{n} \le \delta\]

\[\Pr_{\vec{v}}[(\vec{v} \in S) \text{ or } (\vec{v} \oplus \vec{e_i} \in S)] \le 2 \delta\]

Therefore, \(\Pr_{\vec{v}}[\Dec \text{ is correct}] \ge 1 - 2\delta\)

From the union bound, it follows that the aggregate error is less than \(2\delta\).

\subsection{Systematic Codes}

% Polynomial blowup for codewords, but sublinear local decoding.

As a side note, \(k\) field elements can be construed as a \(\deg \le k-1\) polynomial.
Either via coefficients or via the 1st \(k\) evaluations of the polynomial (and interpolating the rest of the codeword).
This has the advantage of including the original message as the first \(k\) elements of the codeword.
Codes where codewords include the original message are called \emph{systematic codes}.

For a systematic code, local correction implies local decoding.

\paragraph{Formally}

A code \(C : [q]^k \to [q]^n\) is systematic if for every subset \(S \subseteq [n]\) of size \(k\) such that \(\forall \vec{x} \in [q]^k\), the codeword \(C(\vec{x})\) restricted to coordinates in \(S\) is equal to \(\vec{x}\).

\subsection{Reed Solomon as an LDC}

Take some Reed Solomon code with large \(k\). This defines a \(\deg \le k-1\) polynomial:

\[f_{\vec{m}} = \sum_{i=0}^{k-1} m_i x^i\]
\[\vec{m} \to (f_{\vec{m}}(\alpha_1), \ldots, f_{\vec{m}}(\alpha_n))\]

For any \(k-1\) symobls of the codeword (excluding \(\alpha=i\)) gives us no information on \(m_i\).
Any value of \(m_i \in \Fq\) is equally likely.

As a result, Reed Solomon is not a good LDC.

\subsection{Reed-Muller Codes}

Reed-Muller codes are a generalization of Reed-Solomon codes to higher dimensions (multi-variate polynomials).

Effectively, each codeword is a surface in this higher dimension.

\paragraph{Formally}

A Reed-Muller code has 3 parameters: \(q\) - field size, \(m\) dimensions (number of variables), and \(\ell\) degree bound.
The corresponding RM code consists of all evaluations (inputs ranging over \(\Fq^m\)) of all polynomials of total degree bounded by \(\ell\) in \(\Fq[q][x_1,\ldots,x_m]\).

This means that codewords are low degree polynomials.

The block length \(n = q^m\) (all \(\Fq^m\) input evals).

The dimension of this code is \(k = \binom{m+\ell}{\ell}\) which is the number of free coefficients in such a polynomial.
\footnote{Can be thought of as distributing \(\ell\) balls into \(m+1\) bins representing \(x_m,\ldots,x,1\) monomials}

The distance of the code is \(d = \left(1 - \frac{\ell}{q}\right) q^m\)

\paragraph{Schwartz Lemma}

Let \(f \in \Fq[q][x_1, \ldots, x_m]\) be a non-zero polynomial of total degree \(\le \ell < q\).

\[
  \Pr_{(a_1,\ldots,a_m) \gets \Fq^m} [f(a_1,\ldots,a_m) = 0] \le \frac{\ell}{q}
\]

The full proof is by induction on the number of variables.

This is another way of stating the bound on the number of roots.

\subsubsection{Local Correction}

Let \(m, \ell \in \mathbb{N}\) and \(q\) is a prime power.
We claim there exists a linear code of dimension \(k = \binom{m + \ell}{\ell}\) in \(\Fq^n\) where \(n = q^m\) that is a \((\ell+1, \delta, (\ell + 1) \delta)\) locally correctable code for any \(\delta > 0\).

Given oracle access to \(\vec{y} = \text{evals on all of \(\Fq^m\)}\) corrupted from a codeword in at most \(\delta\) fraction of coordinates.
Also given a point \(\vec{w} \in \Fq^m\) which corresponds to the index of the codeword we wish to recover.

Choose a random line through \(\vec{w}\) in \(\Fq^m\) (the floor) for some randomly chosen vector \(\vec{v} \in \Fq^m\)
\[L = \{\vec{w} + \lambda \vec{v} \mid \lambda \in \Fq\}\]

Let \(S\) be an arbitrary subset of \(\Fq \setminus \{0\}\) with \(|S| = \ell + 1\).
This is the number of evaluation points in the line needed to interpolate the polynomial.

Query \(\vec{y}\) at the \(\ell + 1\) coordinates corresponding to these points on the line:

\[
  \{\vec{w} + \lambda \vec{v} \mid \lambda \in S\} \subseteq L
\]

Denote the answers by \(\{e_\lambda\}_{\lambda \in S}\) where \(e_\lambda \in \Fq\) is the evaluation at \(\vec{w} + \lambda \vec{v}\).

Note that the induced cross section (plane as defined by \(L\)) gives us a Reed-Solomon code.

Recover the unique univariate polynomial \(h \in \Fq(x)\) with \(\deg h \le \ell\) such that \(h(\lambda) = e_\lambda \;\forall \lambda \in S\).

Output \(h(0)\).

\paragraph{Proof of \(\varepsilon\)}

If there are no errors in the queried values \(\{e_\lambda\}_{\lambda \in S}\), then we get a correct recovery.

\[
  \Pr_{\vec{v}}[\text{ there is an error in the queries}] \le \sum_{\lambda \in S} \Pr_{\vec{v}}[\text{there is an error not in the eval point}]
\]
% by union bound

Each \(\vec{w} + \lambda \vec{v}\) is uniform over \(\Fq^m\).

Therefore, we have less than \(\le |S| \frac{\text{\# error locations}}{\text{\# total lcoations}}\).

This implies that \(\le (\ell + 1) \delta\) probability of accurate correction.

\section{Next Time}

Applications to cryptography

\end{document}
