\documentclass{idc_msc}

\title{Distributed Algorithms\\\large Lecture 11}
\date{2017-06-21 \\ Last edited \currenttime\ \today}
\author{Lecture by Dr. Gadi Taubenfeld\\Typeset by Steven Karas}

\begin{document}

\maketitle

\paragraph{Disclaimer}

These lecture notes are based on the lecture for the course Distributed Algorithms, taught by Dr. Gadi Taubenfeld at IDC Herzliyah in the spring semester of 2017.
Sections may be based on the lecture slides and accompanying book written by Dr. Gadi Taubenfeld.

\paragraph{Agenda}

\begin{itemize}
  \item Barrier synchronization
\end{itemize}

\section{Barriers}

\paragraph{Definition}
A barrier is a synchronization approach for multiple processes in which execution only continues after all the processes reach the barrier point.

\paragraph{Design goals}

\begin{itemize}
  \item Reusability (this is necessary for correctness)
  \item Symmetric-ness (same amount of work for all processes)
  \item No need for initialization
  \item Shared space complexity
  \item Contention complexity
  \item Time complexity
\end{itemize}

\subsection{Fetch-and-increment Register}

Provides an atomic operation that increments the register and returns the previous value.

Some variants impose a modulo as part of the operation.

\subsection{Await macro}

Spin on a condition.

\subsection{Simple barrier with an atomic counter}

Using a shared fetch and increment and an atomic register, and local copies of each:

\begin{lstlisting}[frame=L,mathescape=true,numbers=left,title={For process i in 0...n}]
local.go = go
local.counter = fetch-and-increment(counter)
if local.counter + 1 = n:
  counter = 0
  go = 1 - 0
else:
  await(local.go != go)
\end{lstlisting}

\subsubsection{An incorrect variant}

This is incorrect because a starved process can get stuck at the barrier and lead to deadlock in a second round through the same barrier.

\begin{lstlisting}[frame=L,mathescape=true,numbers=left,title={For process i in 0...n}]
local.go = go
local.counter = fetch-and-increment(counter)
if local.counter + 1 = n:
  counter = 0
  go = 1 - 0
else:
  await(local.go != go)
\end{lstlisting}

\subsubsection{Reducing shared memory contention}

There is high contention on the go bit.

We can reduce this by replacing the single bit with a set of $n$ bits, where each process $p_i$ spins only on $go[i]$.

In distributed shared memory, this can reduce the remote memory accesses to $n$.

\subsection{Barrier without initialization}

If we are able to initialize variables, we must already have synchronization. This approach assumes no prior synchronization.

\begin{lstlisting}[frame=L,mathescape=true,numbers=left,title={For process i in 0...n}]
local.go = go
local.counter = counter
counter := counter + 1 (mod n)
until local.go != go:
  if counter = local.counter:
    go = 1 - local.go
\end{lstlisting}

Some minor modifications can make this correct for 

\subsection{Test-and-set bit}

Provides two atomic operations:

\textit{test-and-set} sets the bit to 1, and returns the previous value.

\textit{reset} sets the bit to 0.

\paragraph{test-and-test-and-set bit}

Provides an atomic read operation.

\subsection{Test-and-set barrier}

With three shared variables: leader (test-and-set), countflag (test-and-test-and-set), and go (atomic-register).
\marginpar{Steven: The algorithm in the slides is not correct for n=1}

\begin{lstlisting}[frame=L,mathescape=true,numbers=left,title={For process i in 0...n}]
local.go = go
if test-and-set(leader) = 0:
  local.counter = 0
  until local.counter = n - 1:
    await(countflag = 1)
    local.counter += 1
    reset(countflag)
  reset(leader)
  go = 1 - go
else:
  await(test-and-set(countflag) = 0)
  await(local.go != go)
\end{lstlisting}

\subsubsection{Variant without initialization}

The basic approach here is to preelect the leader and have it count each process twice (until the counter is $2n-2$).
This ensures that at most $n-1$ processes have been counted twice, and one has been counted either once or twice.

\begin{lstlisting}[frame=L,mathescape=true,numbers=left,title={For process i in 0...n}]
local.go = go
if i = 1:
  local.counter = 0
  until local.counter = 2n - 2:
    await(countflag = 1)
    local.counter += 1
    reset(countflag)
  reset(leader)
  go = 1 - go
else:
  await(test-and-set(countflag) = 0)
  await(test-and-set(countflag) = 0)
  await(local.go != go)
\end{lstlisting}

Note this algorithm is asymmetric, which we consider to be undesireable.

\subsection{Tree-based barriers}

Organize the processes into a binary tree where each node is owned by a process.

Each node waits for its children, and then reports to its parent.
Once the root gets reports from both its children, it sends a message back down the tree.

Another approach is to run a tournament between nodes until a root is chosen, and then continue.

\begin{lstlisting}[frame=L,mathescape=true,numbers=left,title={For process i in 0...n}]
if i = 1:
  await(arrive[2] = 1); arrive[2] = 0
  await(arrive[3] = 1); arrive[3] = 0
  go[2] = 1; go[3] = 1;
else if i $\le \frac{n-1}{2}$:
  await(arrive[2i] = 1); arrive[2i] = 0
  await(arrive[2i+1] = 1); arrive[2i+1] = 0
  arrive[i] = 1
  await(go[i] = 1); go[i] = 0
  go[2i] = 1; go[2i+1] = 1
else:
  arrive[i] = 1
  await(go[i] = 1); go[i] = 0   
\end{lstlisting}


\subsection{Read-modify-write Registers}

A register $r$ with $n$ bits.
Supports an atomic operation read-modify-write, which does the following atomically:

\begin{lstlisting}[frame=L,mathescape=true,numbers=left]
read-modify-write(r: register, f: function):
  orig_r = r
  r = f(r)
  return orig_r
\end{lstlisting}

\subsection{See-saw barrier}

Presented by \href{https://www.faculty.idc.ac.il/gadi/MyPapers/1996FMRT-wakeup.pdf}{M. J. Fischer, S. Moran, S. Rudich, and G. Taubenfeld in 1996}

3 bit algorithm, but we need 2 RMW bits.

The idea is that each process has 2 tokens, which can either be absorbed or emitted at a time.

The seesaw has one side up and one side down.
Each process that enters the algorithm needs to enter the seesaw on either the left or right side.

The token bit has two states: no-token-present and token-present.

The see-saw bit has two states: left-side-down and right-side-down.

Each round, a process runs one of five rules. Once a process has gathered at least $2n$ tokens, it sets the go bit. A process that has gotten off the seesaw awaits the go bit to flip.

\begin{enumerate}
  \item if the process has never been on; save the go bit locally, get on the up side, and swing the seesaw.
  \item if the process is on the down side, has tokens, and there is no token; emit a token to the shared state, if no tokens left, get off and swing the seesaw.
  \item if the process is on the up side and there is a shared token; take the token.
  \item if the process has at least $2n$ tokens, it gets off the seesaw and flips the go bit.
  \item if the go bit has been flipped, all processes have arrived, and continues past the barrier.
\end{enumerate}

The shared memory is accessed at least $O(n)$ times, and at worst $O(n^2)$.

\section{Next week}

Semaphore barriers.

Basically, a way to snapshot the entire system and get a consistent state.

\end{document}
